%%%%%%%%%%%%%%%%%%%%%%%%%%%%%%%%%%%%%%%%%%%%%%%%%%%%%%%%%%%%%%%%%%%%%%%%%%%%%%%

\chapter{INTRODUCTION}

% Background and Context
% Reserach Problem
% Significance of Study

\section{General objective}

To explore a cold pool parameterization coupled with the Grell-Freitas convection parameterization within the Model for Ocean-laNd-Atmosphere predictioN (MONAN).


\section{Specific objectives}

\begin{enumerate}
    \item Assess the impact of the cold pool parameterization scheme on hurricane forecasted trajectory, intensity and rainfall;
    \item Perform sensitivity experiments within a hurricane case study, investigating the influence of initial conditions, cold pool duration, the displacement of the maximum mass flux height, type of initial condition and resolution;
    \item Identify the optimal parameter configuration for the cold pool parameterization to be used in MONAN;
    \item Evaluate the performance of this optimal configuration in other hurricane case study.
\end{enumerate}

\section{Scientific questions}

This work builds upon the findings and open issues identified in \citeonline{freitas2024parameterization}. The following scientific questions are proposed:

\begin{itemize}
    \item Does the GF-Cold Pool (GF-CP) scheme improve the forecast of hurricane trajectory, intensity, and rainfall in MONAN?
    \item Which parameters within the GF-CP have the most significant influence on hurricane representation in MONAN?
    \item How robust is the optimal GF-CP configuration when applied to a different hurricane event?
    \item Can improvements in cold pool representation lead to better rainfall predictions in tropical cyclones?
\end{itemize}
%% Think maybe in other two scientifics questions

\section{Thesis organization}

% This thesis is structured into seven chapters. Chapter 1 introduces the research, presenting its background, general and specific objectives, guiding scientific questions, and the overall structure of the work. Chapter 2 provides a review of the relevant literature, including the theoretical foundations of convective cloud dynamics, the mass-flux parameterization, cold pools, and the modeling of hurricanes. Chapter 3 describes the observational datasets and outlines the MONAN model configuration used throughout the study. Chapter 4 details the methodology, focusing on the diagnostic metrics and the workflow adopted to evaluate the cold pool parameterization scheme and its impact on hurricane forecasts. Chapter 5 presents a detailed case study of Hurricane Beryl, including a series of sensitivity experiments that lead to the identification of an optimal parameter configuration for the cold pool scheme. Chapter 6 applies this optimal setup to a second case, Hurricane Helene, in order to assess the robustness and generalization of the proposed configuration. Finally, Chapter 7 summarizes the main findings, addresses the initial scientific questions, discusses the scientific contributions and limitations of the work, and outlines recommendations for future research.




