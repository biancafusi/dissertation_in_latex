%%%%%%%%%%%%%%%%%%%%%%%%%%%%%%%%%%%%%%%%%%%%%%%%%%%%%%%%%%%%%%%%%%%%%%%%%%%%%%%

\chapter{LITERATURE REVIEW}
\label{ch:convection}

% Theoretical Framework
% Current State of Knowledge
% Research Gaps
% Key Concepts and Definitions


%General aspects for understanding the Cold Pool (CP) phenomenon and its modeling aspects are presented in this chapter. Starting with the general convective structure, we provide a glimpse of atmospheric convection to characterize better the main features involved with CPs. The development of CPs and interaction features are explained next, focusing on Cloud Resolving Models (CRMs) and observational information. These features are crucial for understanding how CPs can trigger and organize new convection, subsequently upscaling them into Mesoscale Convective Systems (MCSs). Further on, a brief review of the Mass-Flux approach and the CP parameterization is outlined. Some aspects of the CP interacting with the selected study cases are described, to elucidate CP interacting with Hurricanes and their presence over the Amazon Basin. Finally, we conclude with the chosen convective organization metric.

\section{Overview of convective cloud dynamics}

\section{Cold pools}

\section{The mass-flux approach}

\section{The cold pool parameterization}

\section{Hurricanes - a model forecast view}

\section{Key points recall}




